%! Author = Chris
%! Date = 6/13/2022

% Preamble
\documentclass[11pt]{article}

% Packages
\usepackage{amsmath}
\usepackage{tikz}

% Document
\begin{document}

\section*{1}
    \begin{enumerate}
        \item The best sequence is $(2, 1, 3, 4)$ with an objective value of $333$
        \item By putting items with more weight towards the beginning, you minimize the values of $C_j$ they will be multiplied by
        \item By putting the jobs with shorter processing time towards the beginning, you make it so that the average value of $C_j$ is lower
        \item $w_j -p_j = (3, 6, 2, 1)$ giving us order $(2, 1, 3, 4)$. \\
            $\frac{w_j}{p_j}=(2, 2.2, 1.29, 1.25)$ giving us order $(2, 1, 3, 4)$. \\
            Although both give us the same order, doing it by decreasing order of $\frac{w_j}{p_j}$ is optimal.
    \end{enumerate}

\section*{2}
    \begin{enumerate}
        \item The best sequence is $(1, 2, 3, 4)$ with an objective value of $6$
        \item The earlier the due date, the earlier you want to have it done, generally
        \item By putting jobs with smaller processing time earlier, you produce less late jobs overall.
        This doesn't seem like a good metric for reducing $L_{\max}$ though.
        \item Sequencing jobs in $d_j p_j$ order is the most suitable generic rule since it factors both due date and processing time.
    \end{enumerate}

\section*{3}
    \begin{enumerate}
        \item The best sequences are $[(1, 3, 2, 4), (1, 3, 4, 2), (2, 3, 1, 4), (2, 3, 4, 1), (2, 4, 1, 3), (2, 4, 3, 1), (3, 4, 1, 2), (3, 4, 2, 1)]$ with an objective value of $2$
        \item Sequence them in decreasing $d_j - p_j$ order
    \end{enumerate}

\section*{4}
    \begin{enumerate}
        \item The best sequences are $[(1, 2, 4, 3)]$ with an objective value of $22$
        \item Not possible, this is NP hard
    \end{enumerate}

\section*{5}
    \begin{enumerate}
        \item The optimal solution is $(1, 7), (2, 8), (3, 5), (4, 6), (9, 10, 11)$ with an objective value of 15
    \end{enumerate}

\section*{6}
The best sequences are $[(3, 2, 4, 1)]$ with an objective value of $34$

\section*{7}
The best sequences are $[(3, 4, 2, 1)]$ with an objective value of $35$

\section*{8}
The best sequences are $[(3, 4, 2, 1)]$ with an objective value of $35$

\section*{9}
The best schedule does $(3, 2, 4, 1)$ on machine 1 and $(4, 1, 3, 2)$ on machine 2 and the objective value is $30$

\section*{10}
The best schedule does $(2, 1, 4, 3)$ on machine 1 and $(4, 3, 2, 1)$ on machine 2 and the objective value is $30$

\section*{11}
It's easier because you can treat the first case as a special case of the second where you break each job $i$ into $p_i$ different segments with $p=1$ and $r=r_i$

\section*{12}
Each job $i$ will take $a_i$ to setup, $p_i$ to run, and $b_i$ to break down. Therefore the order doesn't matter and the makespan is $C_{\max}=\sum a_i + b_i + p_i$

\section*{13}
With $p=0$, this is just a task of scheduling jobs to reduce the total setup time which is equivalent to TSP

\section*{14}
Can introduce jobs with $w_j=\infty$ that have $r_j$ equal to when the breakdown occurs and $p_j$ equal to the duration of the breakdown

\section*{15}
There are $n$ jobs and $n$ slots and each job belongs in a slot and you are therefore assigning each job to minimize a cost function

\section*{16}
You are assigning each job to one of the $m$ parallel machines and get the same $A$ matrix as you would in the transportation problem

\section*{17}
LHS: The best case is that all $m$ machines have an equal load and the minimum load is $\frac{\sump_j}{m}$
RHS: The worst case is there is

\section*{18}
For all machines $i \notin M_j, v_{ij}=0, p_{ij}=\infty$ which makes $Pm | M_j | \gamma$ a special case of $Rm || \gamma$

\section*{19}
Block means the item must stay in machine 1 until machine 2 is ready and nwt means that machine 2 must be ready right after machine 1.
Both will have the same $C_\max$ as the conditions are equivalent when there is only one set of 2 chained machines.
Can make this $1 | s_{jk} | c_\max$ by using the following $s$, $s_{0j} = p_{1j}, s_{k0} = p_{2k}, s_{jk} = \max(p_{2j}, p_{1k})$


\section*{20}

Om is a relaxation of Fm as you have more options in which order to run the machines.
Therefore the objective in Fm will at best be the same as Om

\section*{21}
If there is one very long job on both machines this will be exceeded.
\begin{center}
\begin{tabular}{ c c c c}
     jobs & 1 & 2 & 3 \\
     \hline
     $p_{1j}$ & 100 & 10 & 5 \\
     $p_{2j}$ & 100 & 10 & 5
\end{tabular}
\end{center}

This would take 200 ro run whn the lower bound is 115


\section*{22}
(i) -> (iii) -> (ii) -> (iv)
        |-> (v) -> (vi)

\section*{23}
Not sure




\end{document}